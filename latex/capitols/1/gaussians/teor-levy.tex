\begin{teo}[Teorema de continuitat de P.\ Lévy]
Considerem una successió $(\mu_n,n\ge 1)$ de probabilitats en $\mathbb{R}$ i denotem per $(\varphi_n,n\ge 1)$ la successió de funcions característiques associades. Aleshores,
\begin{enumerate}[label=\textit{(\roman*)}]
\item Si $\mu_n$ convergeix feblement cap a una probabilitat $\mu$ quan $n\to\infty$, aleshores
\[
\lim_{n\to\infty}\varphi_n(t)=\varphi_\mu(t),\qquad\forall t\in\mathbb{R}.
\]

\item Si $\lim_{n\to\infty}\varphi_n(t)=\varphi(t)$, on $\varphi$ és una funció contínua en el zero, aleshores $\varphi$ és la funció característica d’una probabilitat $\mu$, i
\[
\mathrm{w}\!\text{-}\!\lim_{n\to\infty}\mu_n=\mu.
\]
\end{enumerate}
\end{teo}

\begin{proof}
L’apartat \textit{(i)} és una conseqüència immediata de la definició de convergència feble, ja que les funcions $\cos(tx)$ i $\sin(tx)$ són contínues i fitades.

Per a la demostració de \textit{(ii)} utilitzarem la desigualtat de truncació (Proposició~12.6) i el teorema de convergència dominada (ja que $|\varphi_n(t)|\le 1$). Obtenim així
\[
\mu_n\bigl(\{x:|x|\ge a\}\bigr)\le
\frac{1}{a}\int_{-a}^{a}\!\bigl(1-\varphi_n(t)\bigr)\,dt.
\]
Aquesta darrera expressió convergeix cap a
\[
\frac{1}{a}\int_{-a}^{a}\!\bigl(1-\varphi(t)\bigr)\,dt
\quad\text{quan }n\to\infty.
\]
D’altra banda,
\[
\lim_{a\to 0}\frac{1}{a}\int_{-a}^{a}\!\bigl(1-\varphi(t)\bigr)\,dt
=2\bigl(1-\varphi(0)\bigr)=0,
\]
ja que $\varphi$ és contínua en el zero.

Fixem un $\varepsilon>0$ i sigui $a>0$ tal que
$0\le\frac{1}{a}\int_{-a}^{a}\!\bigl(1-\varphi(t)\bigr)\,dt\le\varepsilon$.
A partir d’un cert $n_0$ tindrem
\[
\mu_n\Bigl(\Bigl[-\frac{2}{a},\frac{2}{a}\Bigr]^{\!\!c}\Bigr)
\le\mu_n\bigl(\{x:|x|\ge 2/a\}\bigr)\le
\frac{1}{a}\int_{-a}^{a}\!\bigl(1-\varphi_n(t)\bigr)\,dt\le\varepsilon.
\]
D’altra banda, per $1\le n\le n_0$ podem trobar $a_n>0$ tals que
$\mu_n\bigl([-a_n,a_n]^c\bigr)\le\varepsilon$.
En conclusió, si $b=\max(a_1,\dots,a_{n_0},2/a)$, es compleix que
\[
\sup_n\mu_n\bigl([-b,b]^c\bigr)\le\varepsilon,
\]
i això ens diu que la successió de probabilitats $(\mu_n,n\ge 1)$ és ajustada.

Tenint en compte el criteri de convergència feble (Teorema~11.9), per a demostrar que $\mu_n$ convergeix feblement cap a $\mu$ només cal comprovar que totes les subsuccessions convergents tenen el mateix límit, igual a $\mu$. Suposem que $(\mu_{n_i},i\ge 1)$ és una subsuccessió convergent cap a una probabilitat $\nu$. Tindrem, per la part~\textit{(i)},
\[
\varphi_\nu(t)=\lim_{i\to\infty}\varphi_{n_i}(t)=\varphi(t).
\]
Això ens diu que $\varphi$ és la funció característica d’una probabilitat (ja que sempre podem trobar una subsuccessió feblement convergent) que designarem per $\mu=\nu$, i totes les parcials convergents tenen límit $\mu$. Acaba així la demostració del teorema.
\end{proof}