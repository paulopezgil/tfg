\begin{teo}
Siguin $\mu\in\mathbb{R}^n$ i $\Sigma$ una matriu simètrica d'ordre $n$ definida no negativa.
Aleshores, existeix una probabilitat a $\mathbb{R}^n$ que té per funció característica
\[
\varphi(t)=\exp\Bigl(i t^T\mu-\tfrac12 t^T\Sigma t\Bigr),\qquad
\forall t\in\mathbb{R}^n.
\]
Anomenarem aquesta mesura la \emph{normal normal n-dimensional} (o \emph{gaussiana n-dimensional})
amb mitjana $\mu$ i matriu de covariància $\Sigma$
i la designarem per $\mathcal{N}(\mu,\Sigma)$.
\begin{prova}
Vegeu \cite{NualartSanz1990}, pàg 126.
\end{prova}
\end{teo}


\begin{defi}
Direm que un vector aleatori $X$ és \emph{gaussià} si
existeixen $\mu\in\mathbb{R}^n$ i una matriu $\Sigma$
simètrica d'ordre $n$ i definida no negativa
tals que $X\sim\mathcal{N}(\mu,\Sigma)$.
\end{defi}