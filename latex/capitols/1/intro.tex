El moviment brownià és un procés estocàstic fonamental en la teoria de la probabilitat i les seves aplicacions.
Observat per primer cop pel botànic Robert Brown el 1828, el moviment brownià modelitza el moviment irregular del pol·len suspès en aigua.

El primer treball quantitatiu es deu a Bachelier, qui l'any 1900 es va interessar en les fluctuacions dels preus de les accions.
L'any 1905, Einstein va derivar la densitat de transició a partir de la teoria molecular-cinètica de la calor.
Posteriorment N. Wiener va dur a terme un tractament matemàtic rigorós durant els anys 1923 - 1924, proporcionant la primera demostració d'existència.
L'obra més profunda d'aquest període és la de P. Lévy, qui durant els anys 1939-1948 va introduir la construcció per interpolació i va estudiar els temps de pas, les trajectòries mostrals i el temps local.

Actualment, l'abast d'aplicació del moviment brownià va molt més enllà de l'estudi de partícules microscòpiques en suspensió
i inclou la modelització de preus d'accions, soroll tèrmic en circuits elèctrics, i pertorbacions aleatòries en sistemes físics, biològics i econòmics.
Concretament, en el camp de les probabilitats, el moviment brownià és un exemple fonamental de procés estocàstic amb trajectòries contínues.
Entre les seves principals aplicacions en aquest àmbit destaquen, per exemple, el teorema de Donsker i el càlcul estocàstic en l'estudi d'equacions diferencials estocàstiques. 
