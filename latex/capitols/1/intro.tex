El moviment brownià és el nom que rep el moviment irregular del pol·len suspès
en aigua, el qual va ser observat pel botànic Robert Brown el 1828. Aquest
moviment aleatori, que ara s'atribueix als cops que les molècules d'aigua donen
al pol·len, resulta en una dispersió o difusió del pol·len a l'aigua. L'abast
d'aplicació del moviment brownià, tal com es defineix aquí, va molt més enllà de
l'estudi de partícules microscòpiques en suspensió i inclou la modelització de
preus d'accions, de soroll tèrmic en circuits elèctrics, de certs comportaments
límit en sistemes de cues i d'inventari, i de pertorbacions aleatòries en una
varietat d'altres sistemes físics, biològics, econòmics i de gestió.
A més, la integració respecte al moviment brownià ens ofereix una representació
unificadora per a una gran classe de martingales i processos de difusió.
Els processos de difusió representats d'aquesta manera mostren una rica connexió
amb la teoria de les equacions en derivades parcials. En particular, a cada
procés d'aquests li correspon una equació parabòlica de segon ordre que governa
les probabilitats de transició del procés. 

El primer treball quantitatiu sobre el moviment brownià es deu a Bachelier
(1900), qui estava interessat en les fluctuacions dels preus de les accions.
Einstein (1905) va derivar la densitat de transició per al moviment brownià a partir de la teoria molecular-cinètica de la calor.
Posteriorment, N. Wiener (1923, 1924a) va dur a terme un tractament matemàtic rigorós i va proporcionar la primera demostració d'existència.

L'obra més profunda d'aquest primer període relativa al moviment brownià és la
de P. Lévy (1939, 1948); va introduir la construcció per interpolació, va
estudiar en detall els temps de pas i altres funcionals relacionats, va
descriure detalladament l'anomenada estructura fina de la trajectòria mostral
típica i va descobrir la noció i les propietats de la mesure du voisinage o
"temps local". El més sorprenent és que va dur a terme aquest programa sense els
conceptes i eines formals de filtracions, temps d'aturada o la propietat forta
de Markov.
