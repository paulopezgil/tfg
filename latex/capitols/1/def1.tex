\begin{defi}[Moviment brownià unidimensional]\label{def:moviment_brownia}
Un \textbf{moviment brownià} (estàndard, unidimensional) és un procés continu i adaptat $B = \{B_t, \mathcal{F}_t; 0 \leq t < \infty\}$, definit en un espai de probabilitat $(\Omega, \mathcal{F}, P)$, amb les propietats següents:
\begin{itemize}
    \item $B_0 = 0$ q.s. (quasi segurament).
    \item Per a $0 \leq s < t$, l'increment $B_t - B_s$ és independent de $\mathcal{F}_s$ i està distribuït normalment amb mitjana zero i variància $t - s$.
\end{itemize}
De vegades parlarem d'un moviment brownià $B = \{B_t, \mathcal{F}_t; 0 \leq t \leq T\}$ en $[0, T]$, per a algun $T > 0$, i el significat d'aquesta terminologia és evident.
\end{defi}