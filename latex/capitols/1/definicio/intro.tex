Comencem aquesta secció recordant algunes definicions i propietats bàsiques
sobre variables aleatòries i distribucions normals.
Recordem que una variable aleatòria $X$ s'anomena \emph{gaussiana}
si existeixen $\mu \in \mathbb{R}$ i $\sigma^2 \geq 0$ tals que
per a tot $B \in \mathcal{B}(\mathbb{R})$,
\begin{equation*}
\P(X \in B) = \frac{1}{\sqrt{2\pi \sigma^2}} \int_B \exp\left(-\frac{(x-\mu)^2}{2\sigma^2}\right) dx.
\end{equation*}
En aquest cas, escrivim $X \sim N(\mu, \sigma^2)$.
Per altra banda, recordem que la funció característica d'una variable aleatòria $X$
és la funció $\varphi_X: \mathbb{R} \to \mathbb{C}$ definida per
\begin{equation*}
\varphi_X(u) \defeq \mathbb{E}\left[e^{iuX}\right], \quad t \in \mathbb{R}.
\end{equation*}
Aquesta funció permet caracteritzar completament la distribució de $X$.