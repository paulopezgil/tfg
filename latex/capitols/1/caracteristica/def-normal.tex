Recordem que una variable aleatòria $X$ s'anomena Gaussiana,
i escrivim $X \sim \mathcal{N}(\mu, \sigma^2)$,
si existeixen $\mu \in \mathbb{R}$ i $\sigma^2 \geq 0$ tals que
\begin{equation*}
\P(X \in B) = \frac{1}{\sqrt{2\pi \sigma^2}} \int_B \exp\left(-\frac{(x-\mu)^2}{2\sigma^2}\right) dx,
\qquad \forall B \in \mathcal{B}(\mathbb{R}).
\end{equation*}