Finalment, cal comentar que també és habitual definir el moviment Brownià com
un procés estocàstic amb trajectòries contínues quasi segurament.
Això és, afegir la següent propietat a la definició \ref{def:brownia}:
\begin{enumerate}
    \setcounter{enumi}{3}
    \item La funció $t \mapsto B_t$ és contínua quasi segurament.
\end{enumerate}
Si més no, a partir de la definició \ref{def:brownia} i utilitzant l'anomenat
\emph{criteri de continuïtat de Kolmogorov}, es pot demostrar l'existència
d'una versió del moviment brownià amb trajectòries contínues quasi segurament
\footnote{Vegeu \cite{Rovira2020}, pàg 41.}.
Quan ens referim al \emph{moviment brownià estàndard},
ens referirem a la versió amb trajectòries contínues quasi segurament.