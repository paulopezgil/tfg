Finalment, cal comentar que també és habitual definir el moviment Brownià com
un procés estocàstic amb trajectòries contínues quasi segurament.
Això és, afegir la següent propietat a la definició \ref{def:brownia}:
\begin{enumerate}
    \setcounter{enumi}{3}
    \item La funció $t \mapsto B_t$ és contínua quasi segurament.
\end{enumerate}
Si més no, a partir de la definició \ref{def:brownia} i utilitzant l'anomenat
\emph{criteri de continuïtat de Kolmogorov}, es pot demostrar l'existència
d'una versió del moviment Brownià amb trajectòries contínues quasi segurament
\footnote{Vegeu \cite{Rovira2020}, pàg 41.}.
Quan ens referim al \emph{moviment Brownià estàndard},
ens referirem a aquesta versió del procés en qüestió.

També cal esmentar que l'existència del moviment Brownià estàndard
és un resultat no trivial que es pot demostrar de vàries maneres.
Per exemple:
\begin{itemize}
    \item A partir de la definició donada anteriorment, obtenint la densitat conjunta del vector $(B_{t_1}, \cdots, B_{t_n})$
          i utilitzant el \emph{Teorema de Kolmogorov} per a l'existència de processos estocàstics.
          Posteriorment, es pot utilitzar el \emph{criteri de continuïtat de Kolmogorov}
          per a obtenir una versió amb trajectòries contínues quasi segurament, tal i com hem esmentat anteriorment.
          \footnote{Vegeu \cite{Rovira2020}, pàg 40-41.}
    \item Definint el moviment Brownià com un procés gaussià
          (és a dir, tal que tots els seus vectors finits són gaussians multidimensionals)
          amb mitjana zero i matriu de covariància donada per $\mathbb{E}(B_sB_t) = min(s,t)$,
          per a posteriorment utilitzar el \emph{Teorema de consistència de Kolmogorov}
          per a provar la seva existència.
          \footnote{Vegeu \cite{Marquez2024}, pàg 105.}
    \item Demostrant l'existència de la \emph{mesura de Wiener},
          que és la llei del moviment Brownià estàndard,
          entès com a funció aleatòria amb valors a $C[0,\infty)$.
          \footnote{Vegeu \cite{Billingsley1999}, pàg. 87.}
\end{itemize}