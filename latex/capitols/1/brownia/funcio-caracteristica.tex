Finalment, donem una proposició que ens serà útil més endavant
quan estudiem la convergència cap al moviment brownià.
Considerarem el cas $n = 2$.
La demostració del cas general és anàlega però la notació és més pesada.

\begin{prop} \label{prop:funcio-caracteristica-brownia}
Sigui $\{B_t : t \geq 0\}$ un moviment brownià estàndard.
Aleshores, per a qualssevol $0 \leq s < t < \infty$,
la funció característica del vector $(B_s, B_t - B_s)$ és
\begin{equation*}
\varphi_{(B_s, B_t - B_s)}(u,v)
= \exp\left( -\frac{1}{2} \left( u^2 s + v^2 (t-s) \right) \right).
\end{equation*}
\begin{prova}
Per la propietat (2) del moviment brownià, sabem que $B_s$ i $B_t - B_s$ són independents.
Per tant,
\begin{equation*}
\varphi_{(B_s, B_t - B_s)}(u,v)
= \mathbb{E}\left[ e^{i(u B_s + v (B_t - B_s))} \right]
= \mathbb{E}\left[ e^{i u B_s} \right] \mathbb{E}\left[ e^{i v (B_t - B_s)} \right]
\end{equation*}
Per la propietat (3) del moviment brownià,
sabem que $B_s \sim \mathcal{N}(0,s)$. Per tant, per la proposició \ref{prop:caracteristica-normal},
\begin{equation*}
\mathbb{E}\left[ e^{i u B_s} \right] = \exp\left( -\frac{1}{2} u^2 s \right).
\end{equation*}
Anàlogament, com $B_t - B_s \sim \mathcal{N}(0,t-s)$, tenim que
\begin{equation*}
\mathbb{E}\left[ e^{i v (B_t - B_s)} \right] = \exp\left( -\frac{1}{2} v^2 (t-s) \right).
\end{equation*}
Per tant,
\[
\mathbb{E}\left[ e^{i(u B_s + v (B_t - B_s))} \right] = \exp\left( -\frac{1}{2} \left( u^2 s + v^2 (t-s) \right) \right),
\]
com volíem demostrar.
\end{prova}
\end{prop}