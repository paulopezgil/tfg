\begin{obs}
Considerem un procés estocàstic $\{X_t \vcentcolon t \in T\}$
amb trajectòries contínues i valors a $\mathbb{R}$,
definit en un espai de probabilitat $(\Omega, \mathbb{F}, \P)$.
Aplicant les definicions anteriors,
\begin{itemize}
\item Fixat $t \in T$, la funció $X_t\vcentcolon\Omega \to \mathbb{R}$ és una variable aleatòria.
\item Fixats $t_1, t_2, \ldots, t_n \in T$, la funció $(X_{t_1}, X_{t_2}, \ldots, X_{t_n}) \vcentcolon \Omega \to \mathbb{R}^n$ definida per
      \[
      (X_{t_1}, X_{t_2}, \ldots, X_{t_n})(\omega) = (X_{t_1}(\omega), X_{t_2}(\omega), \ldots, X_{t_n}(\omega)) \quad \forall \omega \in \Omega
      \]
      és un vector aleatori (que s'anomena vector aleatori de dimensió finita).
\item Fixat $\omega \in \Omega$, la trajectòria $X_{\text{\LARGE .}}(\omega)\vcentcolon T \to \mathbb{R}$
      definida per
      \[
      X_{\text{\LARGE .}}(\omega)(t) = X_t(\omega) \quad \forall t \in T
      \]
      és una funció contínua amb valors reals.
\item En conseqüència, la funció $X\vcentcolon\Omega\to C[0,\infty]$ definida per
      \[
      X(\omega) = X_{\text{\LARGE .}}(\omega) \quad \forall \omega \in \Omega
      \] és una funció aleatòria.
\end{itemize}
\end{obs}