Ara, considerem una variable aleatòria $X$
en un espai de probabilitat $(\Omega, \mathcal{F}, \P)$
amb valors a $(S, \mathcal{S})$.
Recordem que la \emph{llei de} $X$ és la mesura de probabilitat
$\mathcal{L}(X) \in \mathcal{P}(S)$ definida per
\begin{equation*}
\mathcal{L}(X)(B) \defeq \P(X^{-1}(B)) = \P(X \in B) \quad \forall B \in \mathcal{S}.
\end{equation*}

\begin{defi} % 2.1.3
Sigui $\{X_n\}_n$ una successió de variables aleatòries en un espai
amb valors a $(S, \mathcal{S})$.
Sigui $X$ una altra variable aleatòria amb valors a $(S, \mathcal{S})$.
Diem que $\{X_n\}_n$ \emph{convergeix en llei} cap $X$,
i escrivim $X_n \xrightarrow{\mathcal{L}} X$,
si les lleis de les variables aleatòries $X_n$
convergeixen feblement cap a la llei de $X$.
És a dir, si
\begin{equation*}
\mathcal{L}(X_n) \xrightarrow{w} \mathcal{L}(X).
\end{equation*}
\end{defi}