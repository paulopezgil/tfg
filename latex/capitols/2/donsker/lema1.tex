\begin{lema} \label{lema-estabilitat-convergencia-en-llei}
Siguin $\left\{ X^{(n)} \right\}_n$, $\left\{ Y^{(n)} \right\}_n$ i $X$
variables aleatòries amb valors a $(S, \rho)$,
tals que per a tot $n \geq 1$, $X^{(n)}$ i $Y^{(n)}$
estan definides en el mateix espai de probabilitats.
Suposem que $X^{n} \xrightarrow{\mathcal{L}} X$
i que $\rho(X^{n}, Y^{n}) \xrightarrow{P} 0$.
Aleshores, $Y^{n} \xrightarrow{\mathcal{L}} X$.
\textcolor{blue}{\cite{KaratzasShreve1991}, pàg 120}



\begin{prova}
Siguin ${\left\{ (\Omega_n, \mathcal{F}_n, \P_n) \right\}}_n$
els espais de probabilitat on estan definides les variables aleatòries
$X^{(n)}$ i $Y^{(n)}$.
Sigui ${\left\{ (\Omega, \mathcal{F}, \P) \right\}}_n$
un altre espai de probabilitat on està definida la variable aleatòria $X$.
Sigui $f: \mathcal{S} \to \mathbb{R}$ una funció contínua i acotada.
Per la observació \ref{obs-convergencia-en-llei}, és suficient demostrar que
\begin{equation*}
\lim_{n \to \infty} \mathbb{E}_n[f(X^{(n)}) - f(Y^{(n)})] = 0.
\end{equation*}



Sigui $M = \sup_{s \in \mathcal{S}} |f(s)| < \infty$.
Per l'observació \ref{obs-relativament-compacte},
com $X^{(n)} \xrightarrow{\mathcal{L}} X$,
aleshores $\{X^{(n)}\}_n$ és relativament compacta.
Pel teorema de Prohorov, aleshores $\{X^{(n)}\}_n$ és ajustada.
Per tant, per a tot $\varepsilon > 0$, existeix un conjunt compacte
$K_{\varepsilon} \subset \mathcal{S}$ tal que
\begin{equation*}
\P_n(X^{(n)} \in K_{\varepsilon}) \geq 1 - \frac{\varepsilon}{6M}, \quad \forall n \in \mathbb{N}.
\end{equation*}



Com f és continua, existeix $0 < \delta < 1$ tal que $|f(x)-f(y)| < \varepsilon/3$
per a qualssevol $x,y \in K_{\varepsilon}$ tals que $\rho(x,y) < \delta$.
Per altra banda, per hipòtesi $\rho(X^{n}, Y^{n}) \xrightarrow{P} 0$.
Per tant, existeix $n_0 \in \mathbb{N}$ tal que
\begin{equation*}
\P_n(\rho(X^{(n)}, Y^{(n)}) > \delta) < \frac{\varepsilon}{6M}, \quad \forall n \geq n_0.
\end{equation*}



Tot plegat, per a tot $n \geq n_0$,
\begin{align*}
\left|
\int_{\Omega_n} f(X^{(n)}) - f(Y^{(n)}) \, d\P_n
\right|
&\le
\frac{\varepsilon}{3} \P_n\left(X^{(n)} \in K, \rho(X^{(n)}, Y^{(n)}) < \delta\right) \\
&\quad + 2M \, \P_n\left(X^{(n)} \notin K\right) \\
&\quad + 2M \, \P_n\left(\rho(X^{(n)}, Y^{(n)}) > \delta\right) \le \varepsilon.
\end{align*}
\end{prova}
\end{lema}