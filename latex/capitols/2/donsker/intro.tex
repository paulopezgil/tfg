Sigui $\{\xi_n\}_n$ una seqüència de variables aleatòries
independents i idènticament distribuïdes amb esperança 0
i variància $\sigma^2$, $0 < \sigma < \infty$.
Considerem la successió de sumes parcials
$S_0 = 0$ i $S_n = \sum_{i=1}^n \xi_i$ per a $n \geq 1$.
Per a cada $n \in \mathbb{N}$, considerem el procés de sumes parcials
\begin{equation*}
\frac{1}{\sigma \sqrt{n}} S_{\lfloor nt \rfloor}, \quad t \in [0,\infty],
\end{equation*}
i la corresponent interpolació lineal d'aquest procés,
\begin{equation} \label {interpolacio-lineal}
X_t^{(n)} = \frac{1}{\sigma \sqrt{n}} \left( S_{\lfloor nt \rfloor} + (nt - \lfloor nt \rfloor) \xi_{\lfloor nt \rfloor + 1} \right), \quad t \in [0,\infty].
\end{equation}

En aquest capítol, veurem que la successió de processos
$\{X^{(n)}\}_n$ convergeix en llei cap al moviment Brownià estàndard $B$.
Per a demostrar això, necessitarem dos lemes tècnics que ens permetran
establir la convergència en llei de certs vectors aleatòris
cap al moviment brownià.