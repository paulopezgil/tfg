De fet, sota la hipòtesi d'ajustament, podem demostrar el següent resultat, que és encara més fort:

\begin{teo} \label{teo-convergencia-finita}
Sigui $X$ un procés continu i sigui $\left\{ X^{(n)}\right\}_n$ una seqüència ajustada de processos continus
tal que per a qualssevol $0 \leq t_1 \leq ... \leq t_d < \infty$,
\begin{equation*}
\left( X_{t_1}^{(n)},...,X_{t_d}^{(n)} \right) \xrightarrow{\mathcal{L}} \left(X_{t_1},...,X_{t_d} \right).
\end{equation*}
Aleshores, $\left\{X^{(n)}\right\}_n \xrightarrow{\mathcal{L}} X$.
\textcolor{blue}{\cite{KaratzasShreve1991}, teorema 4.15, pàg 65}
\begin{prova}
Volem veure que $\left\{X^{(n)}\right\}_n \xrightarrow{\mathcal{L}} X$.
És a dir, que $\mathcal{L}(X^{(n)}) \xrightarrow{w} \mathcal{L}(X)$,
entenent $\left\{X^{(n)}\right\}_n$ i $X$ com funcions aleatòries
amb valors a $(C[0,\infty], \mathcal{B}(C[0,\infty]))$.

Pel teorema de Prohorov, com $\left\{ X^{(n)}\right\}_n$ és ajustada,
aleshores és relativament compacta.
Per tant, tota subsuccessió $\left\{X^{(n_k)}\right\}_k \subset \left\{X^{(n)}\right\}_n$
té una subsuccessió $\left\{Y^{(n)}\right\}_n \subset \left\{X^{(n_k)}\right\}_k$
tal que $\mathcal{L}(Y^{(n)}) \xrightarrow{w} \P$, per a alguna $\P \in \mathcal{P}(C[0,\infty])$.

Suposem que una altra subsuccessió $\left\{Z^{(n)}\right\}_n$ indueix mesures
a $(C[0,\infty], \mathcal{B}(C[0,\infty]))$ que convergeixen feblement a una mesura $\Q \in \mathcal{P}(C[0,\infty])$.
Aleshores, $\P$ i $\Q$ han de tenir les mateixes distribucions a dimensió finita.
És a dir,
\begin{equation*}
\P\left[ w \in C[0,\infty] : (w(t_1), \ldots, w(t_d)) \in A\right]
=
\Q\left[ w \in C[0,\infty] : (w(t_1), \ldots, w(t_d)) \in A\right]
\end{equation*}
per a tot $0 \leq t_1 < t_2 \ldots < t_d < \infty$, $A \in \mathcal{B}(\mathcal{R}^d)$, $d \geq 1$.
Per tant, $\P = \Q$.

Ara, suposem que la seqüència de mesures
$\left\{\mathcal{L}(X^{(n)})\right\}_n$
no convergeix feblement a $\mathcal{L}(X)$.
Aleshores, existeix una funció $f: C[0,\infty] \rightarrow \mathbb{R}$
contínua i acotada tal que el límit\\
$\lim_{n \to \infty} \int f(w) \mathcal{L}(X_n)(dw)$
no existeix, o bé és diferent de $\int f(w) \mathcal{L}(X)(dw)$.
En qualsevol cas, pel teorema de Prohorov, podem escollir una subseqüència
$\left\{\mathcal{L}(X^{(n_k)})\right\}_k$ tal que
$\lim_{k \to \infty} \int f(w) \mathcal{L}(X_k)(dw)$
existeix però és diferent de $\int f(w) \mathcal{L}(X)(dw)$.
Aquesta subseqüència no conté cap subseqüència $\left\{\mathcal{L}(X^{({n_k}_l)})\right\}_l$
tal que $\mathcal{L}(X^{({n_k}_l)}) \xrightarrow{w} \mathcal{L}(X)$,
cosa que contradiu la conclusió del paràgraf anterior.
\end{prova}
\end{teo}