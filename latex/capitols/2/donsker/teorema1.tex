\begin{teo} \label{teo-convergencia-interpolacio}
Sigui $\{X^{(n)}\}_n$ la successió de processos
definits a l'equació \eqref{interpolacio-lineal}.
Siguin $0 \leq t_1 < t_2 < \ldots < t_d < \infty$.
Aleshores,
\begin{equation*}
\left( X_{t_1}^{(n)}, X_{t_2}^{(n)}, \ldots, X_{t_d}^{(n)} \right)
\xrightarrow{\mathcal{L}}
\left( B_{t_1}, B_{t_2}, \ldots, B_{t_d} \right)
\quad \text{quan } n \to \infty,
\end{equation*}
on $\left\{ B_t : t \geq 0 \right\}$ és un moviment Brownià estàndard.






\begin{prova}
Considerarem el cas $d=2$. El cas general és anàleg però la notació és més feixuga.
Siguin $s=t_1$ i $t=t_2$. Volem veure que
\[
\big(X^{(n)}_s, X^{(n)}_t\big) \xrightarrow{\mathcal{L}} \big(B_s, B_t\big).
\]

Per la definició d'interpolació lineal \eqref{interpolacio-lineal}, per a tot $u\ge 0$ tenim
\[
\left| X^{(n)}_u - \frac{1}{\sigma\sqrt{n}} S_{\lfloor nu\rfloor} \right|
\le \frac{1}{\sigma\sqrt{n}}\, \big|\xi_{\lfloor nu\rfloor+1}\big|.
\]
Ara, sigui $\varepsilon>0$. Aplicant la desigualtat de Txebixev,
\[
\P\!\left( \left| X^{(n)}_u - \frac{1}{\sigma\sqrt{n}} S_{\lfloor nu\rfloor} \right| > \varepsilon \right)
\le \P\!\left( \frac{1}{\sigma\sqrt{n}}\, \big|\xi_{\lfloor nu\rfloor+1}\big| > \varepsilon \right)
\le \frac{\sigma^2}{\varepsilon^2\,\sigma^2 n} = \frac{1}{\varepsilon^2 n} \xrightarrow{n\to\infty} 0.
\]
En particular, això val per $u=s$ i $u=t$, i per tant
\begin{equation} \label{condicio1}
\left\| \big( X^{(n)}_s, X^{(n)}_t\big) - \frac{1}{\sigma\sqrt{n}} \left( S_{\lfloor ns\rfloor},\, S_{\lfloor nt\rfloor} \right) \right\|
\xrightarrow{P} 0.
\end{equation}



Per altra banda, siguin $u,v\in\mathbb{R}$.
Aleshores, com les variables aleatòries $\{\xi_n\}_n$ són independents per construcció,

\begin{multline} \label{eq-func-caracteristica}
\lim_{n \to \infty} \mathbb{E}\left[\exp\!\left\{ \frac{iu}{\sigma\sqrt{n}} \sum_{j=1}^{\lfloor ns\rfloor} \xi_j + \frac{iv}{\sigma\sqrt{n}} \sum_{j=\lfloor ns\rfloor+1}^{\lfloor nt\rfloor} \xi_j \right\}\right]=\\
=\mathbb{E}\left[\exp\!\left\{ \frac{iu}{\sigma\sqrt{n}} \sum_{j=1}^{\lfloor ns\rfloor} \xi_j \right\}\right]
\cdot \mathbb{E}\left[\exp\!\left\{ \frac{iv}{\sigma\sqrt{n}} \sum_{j=\lfloor ns\rfloor+1}^{\lfloor nt\rfloor} \xi_j \right\}\right].
\end{multline}







Fixem-nos en el primer factor (l'altre és idèntic substituïnt $s$ per $t-s$). Com que
\[
\left|
\frac{1}{\sigma\sqrt{n}} \sum_{j=1}^{\lfloor ns\rfloor} \xi_j
-
\frac{\sqrt{s}}{\sigma\sqrt{\lfloor ns\rfloor}} \sum_{j=1}^{\lfloor ns\rfloor} \xi_j
\right|
\xrightarrow[n\to\infty]{P} 0,
\]
i pel teorema central del límit
\[
\frac{\sqrt{s}}{\sigma\sqrt{\lfloor ns\rfloor}} \sum_{j=1}^{\lfloor ns\rfloor} \xi_j \xrightarrow{\mathcal{L}} \mathcal{N}(0,s),
\]
aleshores obtenim que
\[
\lim_{n \to \infty}\mathbb{E}\left[\exp\!\left\{ \frac{iu}{\sigma\sqrt{n}} \sum_{j=1}^{\lfloor ns\rfloor} \xi_j \right\}\right]
= e^{-\frac{u^2 s}{2}}.
\]
De manera anàloga,
\[
\lim_{n \to \infty}\mathbb{E}\left[\exp\!\left\{ \frac{iv}{\sigma\sqrt{n}} \sum_{j=\lfloor ns\rfloor+1}^{\lfloor nt\rfloor} \xi_j \right\}\right]
= e^{-\frac{v^2 (t-s)}{2}}.
\]






Substituïnt aquestes expressions a \eqref{eq-func-caracteristica}, obtenim
\[
\lim_{n \to \infty} \mathbb{E}\left[\exp\!\left\{ \frac{iu}{\sigma\sqrt{n}} \sum_{j=1}^{\lfloor ns\rfloor} \xi_j + \frac{iv}{\sigma\sqrt{n}} \sum_{j=\lfloor ns\rfloor+1}^{\lfloor nt\rfloor} \xi_j \right\}\right]
= e^{-\frac{u^2 s}{2}}\, e^{-\frac{v^2 (t-s)}{2}}.
\]

Com hem vist a la proposició \ref{prop:funcio-caracteristica-brownia}, 
aquesta és la funció característica del vector aleatori $(B_s, B_t-B_s)$.
Per tant, pel Teorema de continuïtat de Paul Lévy, tenim que
\[
\left( \frac{1}{\sigma\sqrt{n}} \sum_{j=1}^{\lfloor ns\rfloor} \xi_j,\, \frac{1}{\sigma\sqrt{n}} \sum_{j=\lfloor ns\rfloor+1}^{\lfloor nt\rfloor} \xi_j \right)
\xrightarrow{\mathcal{L}} (B_s, B_t-B_s).
\]

Considerem l'aplicació contínua $\varphi(x,y)=(x, x+y)$.
Pel lema \ref{lema-mapa-continua},
la condició anterior implica que
\begin{equation} \label{condicio2}
\left( \frac{1}{\sigma\sqrt{n}} S_{\lfloor ns\rfloor},\, \frac{1}{\sigma\sqrt{n}} S_{\lfloor nt\rfloor} \right)
\xrightarrow{\mathcal{L}} (B_s, B_t).
\end{equation}

Per tant, per les condicions \eqref{condicio1} i \eqref{condicio2},
i pel lema \ref{lema-estabilitat-convergencia-en-llei},
tenim que
\[\big(X^{(n)}_s, X^{(n)}_t\big) \xrightarrow{\mathcal{L}} \big(B_s, B_t\big),\]
com volíem demostrar.

\end{prova}
\end{teo}