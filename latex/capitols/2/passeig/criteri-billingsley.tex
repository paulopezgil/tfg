Donats els resultats de la secció anterior, només queda per provar
l'ajustament de la successió de processos $\{X^{(n)}\}_n$.
Per a aquest propòsit, ens serà útil el següent criteri:

\begin{teo}[Criteri de Billingsley]
Una successió de processos $\{X^{(n)}\}_n$ és ajustada si se satisfan aquestes dues condicions:
\begin{enumerate}[label=\alph*)]
    \item La successió $\{X^{(n)}_0\}_n$ és ajustada.
    \item Existeixen constants $\gamma \geq 0$ i $\alpha > 1$
          i una funció $F \in \mathcal{C}$ no decreixent tal que
    \begin{equation} \label{eq:billingsley-condition}
    \P\{|X^{(n)}_t - X^{(n)}_s| \geq \lambda\} \leq \frac{1}{\lambda^\gamma} |F(t) - F(s)|^\alpha
    \end{equation}
    per a tot $s, t \in \mathbb{R}$, $n \in \mathbb{N}$ i $\lambda > 0$.
\end{enumerate}
\begin{prova}
Aquest resultat va ser demostrat per primer cop a \cite{Billingsley1968}.
\end{prova}
\end{teo}

\noindent D'una banda, observem que la condició
\[
\mathbb{E}[|X^{(n)}_t - X^{(n)}_s|^\gamma] \leq |F(t) - F(s)|^\alpha
\]
implica \eqref{eq:billingsley-condition} per la desigualtat de Txebixev.
D'altra banda, en el nostre cas, $X^{(n)}_0 = 0$ per a tot $n \in \mathbb{N}$.
Per tant, per a demostrar l'ajustament, és suficient
provar que existeixen constants $\gamma \geq 0$ i $\alpha > 1$
i una funció contínua $F \in \mathcal{C}$ no decreixent tal que
\[
\mathbb{E}[|X^{(n)}_t - X^{(n)}_s|^\gamma] \leq |F(t) - F(s)|^\alpha.
\]