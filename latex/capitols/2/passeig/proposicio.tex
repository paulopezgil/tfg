
Concretament, en el nostre cas on $\{X^{(n)}\}_n$ eren les trajectòries del passeig
aleatori convenientment normalitzades, demostrarem que

\begin{prop} \label{prop-ajustament-passeig}
Per a tot $s < t < \infty$,
\[
\mathbb{E}[(X^{(n)}_t - X^{(n)}_s)^4] \leq C(t - s)^2.
\]
\begin{prova}
Recordem que en el cas del passeig aleatori, $\mathbb{E}[\xi_k] = 0$ i $\operatorname{Var}(\xi_k) = 1$.
Observem que podem escriure els processos $X^{(n)}_t$ de la següent manera:
\[
X^{(n)}_t = \frac{1}{\sqrt{n}} \int_0^{nt} \theta(x)\,dx,
\qquad \text{on} \qquad
\theta(x) = \sum_{k=1}^\infty \xi_k \mathbf{1}_{[k-1,k)}(x).
\]
En efecte, partint de la definició \eqref{interpolacio-lineal}, tenim que
\begin{align*}
X^{(n)}_t
&= \frac{1}{\sqrt{n}} \left( \sum_{j=1}^{\lfloor nt \rfloor} \xi_j + (nt - \lfloor nt \rfloor) \xi_{\lfloor nt \rfloor+1} \right) \\
&= \frac{1}{\sqrt{n}} \left( \sum_{j=1}^{\lfloor nt \rfloor} \int_{j-1}^{j} \xi_j\,dx
+ \int_{\lfloor nt \rfloor}^{nt} \xi_{\lfloor nt \rfloor+1}\,dx \right) \\
&= \frac{1}{\sqrt{n}} \left( \int_0^{\lfloor nt \rfloor} \sum_{k=1}^\infty \xi_k \mathbf{1}_{[k-1,k)}(x)\,dx
+ \int_{\lfloor nt \rfloor}^{nt} \sum_{k=1}^\infty \xi_k \mathbf{1}_{[k-1,k)}(x)\,dx \right) \\
&= \frac{1}{\sqrt{n}} \int_0^{nt} \sum_{k=1}^\infty \xi_k \mathbf{1}_{[k-1,k)}(x)\,dx.
\end{align*}

Aleshores,
\begin{equation} \label{eq:moment-quart}
\begin{split}
\mathbb{E}\bigl[(X^{(n)}_t - X^{(n)}_s)^4\bigr]
&= \frac{1}{n^2} \mathbb{E}\left[\left(\int_{ns}^{nt} \theta(x)\,dx\right)^4\right] \\
&= \frac{1}{n^2} \mathbb{E}\left[\int_{ns}^{nt} \cdots \int_{ns}^{nt} \theta(x_1) \theta(x_2) \theta(x_3) \theta(x_4)\,dx_1 \cdots dx_4\right] \\
&= \frac{24}{n^2} \mathbb{E}\left[\int_{ns}^{nt} \cdots \int_{ns}^{nt} \theta(x_1) \theta(x_2) \theta(x_3) \theta(x_4)
\mathbf{1}_{\{x_1 \leq x_2 \leq x_3 \leq x_4\}}\,dx_1 \cdots dx_4\right] \\
&= \frac{24}{n^2} \int_{ns}^{nt} \cdots \int_{ns}^{nt}
\mathbb{E}\bigl[\theta(x_1) \theta(x_2) \theta(x_3) \theta(x_4) \mathbf{1}_{\{x_1 \leq x_2 \leq x_3 \leq x_4\}}\bigr]
\,dx_1 \cdots dx_4.
\end{split}
\end{equation}

Observem que
\begin{align*}
&\mathbb{E}\bigl[\theta(x_1) \theta(x_2) \theta(x_3) \theta(x_4) \mathbf{1}_{\{x_1 \leq x_2 \leq x_3 \leq x_4\}}\bigr] \\
&\qquad = \mathbb{E}\left[\sum_{k_1,k_2,k_3,k_4=1}^\infty \xi_{k_1} \xi_{k_2} \xi_{k_3} \xi_{k_4}
\mathbf{1}_{[k_1-1,k_1)}(x_1) \cdots \mathbf{1}_{[k_4-1,k_4)}(x_4) \mathbf{1}_{\{x_1 \leq x_2 \leq x_3 \leq x_4\}}\right] \\
&\qquad = \sum_{k_1,k_2,k_3,k_4=1}^\infty \mathbb{E}[\xi_{k_1} \xi_{k_2} \xi_{k_3} \xi_{k_4}]
\mathbf{1}_{[k_1-1,k_1)}(x_1) \cdots \mathbf{1}_{[k_4-1,k_4)}(x_4) \mathbf{1}_{\{x_1 \leq x_2 \leq x_3 \leq x_4\}}.
\end{align*}

D'altra banda, per la independència i $\mathbb{E}[\xi_k] = 0$ s'obté
\[
\mathbb{E}[\xi_{k_1} \xi_{k_2} \xi_{k_3} \xi_{k_4}] =
\begin{cases}
0, & \text{si existeix } j \text{ tal que } k_j \neq k_i\ \forall i \neq j, \\
1, & \text{si } k_1 = k_2 \text{ i } k_3 = k_4 \text{ (i } k_1 \neq k_3\text{)}, \\
1, & \text{si } k_1 = \cdots = k_4.
\end{cases}
\]
Observem que en el cas del passeig aleatori, $\xi_k^4 = 1$ sempre, ja que $\xi_k \in \{-1, 1\}$.
Per tant només contribueixen els termes amb parelles iguals o tots iguals, i l'expressió anterior es redueix a
\[
\sum_{k,j=1}^\infty \mathbf{1}_{[k-1,k)}(x_1) \mathbf{1}_{[k-1,k)}(x_2)
\mathbf{1}_{[j-1,j)}(x_3) \mathbf{1}_{[j-1,j)}(x_4) \mathbf{1}_{\{x_1 \leq x_2 \leq x_3 \leq x_4\}}
+ O(1).
\]

Substituint a \eqref{eq:moment-quart} i descartant els termes d'ordre inferior, obtenim
\[
\frac{24}{n^2} \int_{ns}^{nt} \cdots \int_{ns}^{nt} \sum_{k,j=1}^\infty
\mathbf{1}_{[k-1,k)}(x_1) \mathbf{1}_{[k-1,k)}(x_2) \mathbf{1}_{[j-1,j)}(x_3) \mathbf{1}_{[j-1,j)}(x_4)
\mathbf{1}_{\{x_1 \leq x_2 \leq x_3 \leq x_4\}}\,dx_1 \cdots dx_4.
\]
Utilitzant la desigualtat
$\mathbf{1}_{\{x_1 \leq x_2 \leq x_3 \leq x_4\}} \leq \mathbf{1}_{\{x_1 \leq x_2\}} \mathbf{1}_{\{x_3 \leq x_4\}}$
i separant les sumes i integrals, s'obté
\[
\frac{24}{n^2} \left( \sum_{k=1}^\infty \int_{ns}^{nt} \int_{ns}^{nt}
\mathbf{1}_{[k-1,k)}(x_1) \mathbf{1}_{[k-1,k)}(x_2) \mathbf{1}_{\{x_1 \leq x_2\}}\,dx_1 dx_2 \right)^2.
\]

Ara fem el canvi de variables $y_1 = x_1/n$ i $y_2 = x_2/n$. Així la quantitat anterior esdevé
\[
\frac{24}{n^2} n^4 \left( \int_s^t \int_s^t \sum_{k=1}^\infty
\mathbf{1}_{[\frac{k-1}{n},\frac{k}{n})}(y_1) \mathbf{1}_{[\frac{k-1}{n},\frac{k}{n})}(y_2)
\mathbf{1}_{\{y_1 \leq y_2\}}\,dy_1 dy_2 \right)^2.
\]
Observem que
\[
\sum_{k=1}^\infty \mathbf{1}_{\{\frac{k-1}{n} \leq y_1 < \frac{k}{n}\}}
\mathbf{1}_{\{\frac{k-1}{n} \leq y_2 < \frac{k}{n}\}}
\leq \mathbf{1}_{\{y_2 - y_1 \leq \frac{1}{n}\}}.
\]
Per tant la nostra expressió està acotada per
\[
24 n^2 \left( \int_s^t \int_s^t \mathbf{1}_{\{y_2 - y_1 \leq \frac{1}{n}\}} \mathbf{1}_{\{y_1 \leq y_2\}}\,dy_1 dy_2 \right)^2.
\]
Calculant l'integral,
\begin{align*}
24 n^2 \left( \int_s^t \int_{\max(y_2 - \frac{1}{n}, s)}^{y_2} dy_1\,dy_2 \right)^2
&\leq 24 n^2 \left( \int_s^t \int_{y_2 - \frac{1}{n}}^{y_2} dy_1\,dy_2 \right)^2 \\
&= 24 n^2 \left( \int_s^t \frac{1}{n}\,dy_2 \right)^2
= 24 (t - s)^2.
\end{align*}
Prenent $C = 24$, hem provat la desigualtat desitjada.
\end{prova}
\end{prop}

\medskip

Ja hem vist a la secció anterior (teorema \ref{teo-convergencia-interpolacio})
que les distribucions finito-dimensionals del procés $\{X^{(n)}\}_n$
convergeixen en llei cap a les del moviment brownià.
Per completar la demostració de la convergència en llei del procés,
només ens calia verificar la condició d’ajustament de la successió $\{X^{(n)}\}_n$.
Mitjançant la proposició \ref{prop-ajustament-passeig} i el criteri de Billingsley,
hem provat que aquesta condició es compleix en el cas del passeig aleatori.
Per tant, pel teorema \ref{teo-convergencia-finita}, el passeig aleatori
convergeix en llei cap al moviment brownià.