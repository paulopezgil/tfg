\section{L'espai $C[0, \infty]$}

En el cas particular del moviment brownià, és útil treballar amb l'espai $C[0, \infty)$
de funcions contínues definides a l'interval $[0, \infty)$ i amb valors reals.




\begin{prop}
La funció $\rho : C[0, \infty) \rightarrow [0, \infty)$ definida per
\begin{equation*}
\rho(f,g) \defeq \sum_{n=1}^{\infty} \frac{1}{2^n} \max_{0 \leq t \leq n}\left( \min\left( |f(t) - g(t)|, 1 \right) \right).
\end{equation*}
és una mètrica a $C[0, \infty)$.

\begin{prova}
\textcolor{blue}{\cite{KaratzasShreve1991}, exercici 4.1}
\end{prova} 
\end{prop}


A continuació, donarem una caracterització de l'ajustament en l'espai $C[0, \infty)$.
Per a fer-ho, necessitarem la noció de \emph{mòdul de continuïtat} i el \emph{Teorema d'Arzelà-Ascoli}.

\begin{defi}
Sigui $T > 0$.
El \emph{mòdul de continuïtat} a $[0,T]$ és la funció
$m^T: C[0, \infty) \times [0, \infty) \rightarrow [0, \infty)$
definida per
\begin{equation*}
m^T(f, \delta) \defeq \sup_{\substack{|s-t| \leq \delta \\ 0 \leq s,t \leq T}} |f(s) - f(t)|, \quad \delta \geq 0.
\end{equation*}
\end{defi}

\begin{teo} (Teorema d'Arzelà-Ascoli)
Un conjunt $A \subset C[0, \infty)$ és relativament compacte si i només si
\begin{equation*}
\sup_{f \in A} |f(0)| < \infty
\end{equation*}
i a més, per a tot $T > 0$,
\begin{equation*}
\lim_{\delta \to 0} \sup_{f \in A} m^T(f, \delta) = 0.
\end{equation*}

\prova \textcolor{blue}{\cite{KaratzasShreve1991}, pàg 62}, \textcolor{blue}{\cite{Billingsley1999}, pàg 81}
\end{teo}



\begin{teo}
Una seqüència de mesures de probabilitat $\{\P_n\}_n \subset \mathcal{P}(C[0, \infty))$
és ajustada si i només si
\begin{equation*}
\lim_{\lambda \uparrow \infty} \sup_{n \geq 1} \P_n\left[ f : |f(0)| \geq \lambda \right] = 0,
\end{equation*}
i per a tot $T > 0$, $\epsilon > 0$,
\begin{equation*}
\lim_{\delta \downarrow 0} \sup_{n \geq 1} \P_n\left[ f : m^T(f, \delta) \geq \epsilon \right] = 0.
\end{equation*}

\begin{prova}
\textcolor{blue}{\cite{KaratzasShreve1991}, pàg 63}
\end{prova}
\end{teo}