\section{El principi d'invariància de Donsker}

Sigui $\{\xi_n\}_n$ una seqüència de variables aleatòries
independents i idènticament distribuïdes amb esperança 0
i variància $\sigma^2$, $0 < \sigma < \infty$.
Considerem la successió de sumes parcials
$S_0 = 0$ i $S_n = \sum_{i=1}^n \xi_i$ per a $n \geq 1$.
Per a cada $n \in \mathbb{N}$, considerem el procés de sumes parcials
\begin{equation*}
\frac{1}{\sigma \sqrt{n}} S_{\lfloor nt \rfloor}, \quad t \in [0,\infty],
\end{equation*}
i la corresponent interpolació lineal d'aquest procés,
\begin{equation} \label {interpolacio-lineal}
X_t^{(n)} = \frac{1}{\sigma \sqrt{n}} \left( S_{\lfloor nt \rfloor} + (nt - \lfloor nt \rfloor) \xi_{\lfloor nt \rfloor + 1} \right), \quad t \in [0,\infty].
\end{equation}

En aquest capítol, demostrarem que la successió de processos
$\{X^{(n)}\}_n$ convergeix en llei cap al moviment Brownià estàndard $B$.

% 2.3.1
\begin{lema} \label{lema-estabilitat-convergencia-en-llei}
Siguin $\left\{ X^{(n)} \right\}_n$, $\left\{ Y^{(n)} \right\}_n$ i $X$
variables aleatòries amb valors a $(\mathcal{S}, \rho)$,
tals que per a tot $n \geq 1$, $X^{(n)}$ i $Y^{(n)}$
estan definides en el mateix espai de probabilitats.
Suposem que $X^{n} \xrightarrow{\mathcal{L}} X$
i que $\rho(X^{n}, Y^{n}) \xrightarrow{P} 0$.
Aleshores, $Y^{n} \xrightarrow{\mathcal{L}} X$.
\begin{prova}
Siguin ${\left\{ (\Omega_n, \mathcal{F}_n, \P_n) \right\}}_n$
els espais de probabilitat on estan definides les variables aleatòries
$X^{(n)}$ i $Y^{(n)}$.
Sigui ${\left\{ (\Omega, \mathcal{F}, \P) \right\}}_n$
un altre espai de probabilitat on està definida la variable aleatòria $X$.
Sigui $f: \mathcal{S} \to \mathbb{R}$ una funció contínua i acotada.

Per la observació \ref{obs-convergencia-en-llei}, és suficient demostrar que
\begin{equation*}
\lim_{n \to \infty} \mathbb{E}_n[f(X^{(n)}) - f(Y^{(n)})] = 0.
\end{equation*}

Sigui $M = \sup_{s \in \mathcal{S}} |f(s)| < \infty$.
Per la observació \ref{obs-relativament-compacte},
com $X^{n} \xrightarrow{\mathcal{L}} X$,
aleshores ${X^{n}}_n$ és relativament compacte.
Per el teorema de Prohorov, aleshores és ajustada.
Per tant, per a tot $\varepsilon > 0$, existeix un conjunt compacte
$K_{\varepsilon} \subset \mathcal{S}$ tal que
\begin{equation*}
\P_n(X^{(n)} \in K_{\varepsilon}) \geq 1 - \frac{\varepsilon}{6M}, \quad \forall n \in \mathbb{N}.
\end{equation*}
Com f és continua, existeix $0 < \delta < 1$ tal que $|f(x)-f(y)| < \varepsilon$
per a qualssevol $x,y \in K_{\varepsilon}$ tals que $\rho(x,y) < \delta$.
Sigui $n_0 \in \mathbb{N}$ tal que
\begin{equation*}
\P_n(\rho(X^{(n)}, Y^{(n)}) > \delta) < \frac{\varepsilon}{6M}, \quad \forall n \geq n_0.
\end{equation*}
Aleshores, per a tot $n \geq n_0$,
\begin{align*}
\left|
\int_{\Omega_n} f(X^{(n)}) - f(Y^{(n)}) \, d\P_n
\right|
&\le
\frac{\varepsilon}{3} \P_n\left(X^{(n)} \in K, \rho(X^{(n)}, Y^{(n)}) < \delta\right) \\
&\quad + 2M \, \P_n\left(X^{(n)} \notin K\right) \\
&\quad + 2M \, \P_n\left(\rho(X^{(n)}, Y^{(n)}) \ge \delta\right) \le \varepsilon.
\end{align*}
\end{prova}
\end{lema}

% 2.3.2
\begin{lema} \label{lema-mapa-continua}
Siguin $\left\{ X^{n} \right\}_n$ i $X$ variables aleatòries
amb valors a un espai mètric $\left(S_1, \rho_1\right)$.
Sigui $\left(S_2, \rho_2\right)$ un altre espai mètric
i sigui $f: S_1 \to S_2$ una funció contínua.
Si $X^{n} \xrightarrow{\mathcal{L}} X$, aleshores
$f(X^{n}) \xrightarrow{\mathcal{L}} f(X)$.
\begin{prova}
\textcolor{blue}{\cite{KaratzasShreve1991}, problema 4.5}
\end{prova}
\end{lema}


\begin{teo} \label{teo-convergencia-interpolacio}
Sigui $\left\{X^{(n)}\right\}_n$ la successió de processos
definits a l'equació \eqref{interpolacio-lineal}.
Siguin $0 \leq t_1 < t_2 < \ldots < t_d < \infty$.
Aleshores,
\begin{equation*}
\left( X_{t_1}^{(n)}, X_{t_2}^{(n)}, \ldots, X_{t_d}^{(n)} \right)
\xrightarrow{\mathcal{L}}
\left( B_{t_1}, B_{t_2}, \ldots, B_{t_d} \right)
\quad \text{quan } n \to \infty,
\end{equation*}
on $\left\{ B_t, \mathcal{F}_t^B; t \geq 0 \right\}$ és un moviment Brownià estàndard.
\begin{prova}
\textcolor{blue}{\cite{KaratzasShreve1991}, teorema 4.15, pàg 67}
\end{prova}
\end{teo}

\begin{teo} \label{teo-convergencia-finita}
Sigui $X$ un procés continu i sigui $\left\{ X^{(n)}\right\}_n$ una seqüència ajustada de processos continus
tal que per a qualssevol $0 \leq t_1 \leq ... \leq t_d \leq \infty$,
\begin{equation*}
\left( X_{t_1}^{(n)},...,X_{t_d}^{(n)} \right) \xrightarrow{\mathcal{L}} \left(X_{t_1},...,X_{t_d} \right).
\end{equation*}
Aleshores, $\left\{X^{(n)}\right\}_n \xrightarrow{\mathcal{L}} X$.
\begin{prova}
\textcolor{blue}{\cite{KaratzasShreve1991}, teorema 4.15, pàg 65}
\end{prova}
\end{teo}

\begin{teo} (Teorema de Donsker)
Sigui $\left\{X^{(n)}\right\}_n$ la successió de processos
definits a l'equació \eqref{interpolacio-lineal}.
Aleshores, $X^{(n)} \xrightarrow{\mathcal{L}} B$,
on $\left\{ B_t, \mathcal{F}_t^B; t \geq 0 \right\}$ és un moviment Brownià estàndard.
\begin{prova}
\textcolor{blue}{És conseqüència dels teoremes \ref{teo-convergencia-interpolacio}
i \ref{teo-convergencia-finita}. A més, cal demostrar l'ajustament de la successió $\{X^{(n)}\}_n$,
que es fa servir en el teorema \ref{teo-convergencia-finita}.
El mètode per a demostrar l'ajustament és diferent a \cite{KaratzasShreve1991}
i \cite{Billingsley1999}. M'agradaria comentar-ho més endavant.}
\end{prova}

\end{teo}