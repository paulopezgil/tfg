\begin{defi} \label{def-metrica-prokhorov}
La mètrica de Prokhorov en $\mathcal{P}(S)$ és la mètrica definida per
\begin{equation*}
\pi(\P, \Q) \defeq \inf \left\{ \varepsilon > 0 : \P(A) \leq \Q(A^{\varepsilon}) + \varepsilon \text{ i } \Q(A) \leq \P(A^{\varepsilon}) + \varepsilon, \forall A \in \mathcal{B}(S) \right\},
\end{equation*}
on $A^{\varepsilon} \defeq \{ x \in S : \rho(x, A) < \varepsilon \} = \bigcup_{p \in A} B_\varepsilon(p)$
\end{defi}

\begin{lema} \label{prop-metrica-prokhorov}
Si $(S, \rho)$ és un espai mètric separable, aleshores la convergència feble
de mesures de probabilitat en $\mathcal{P}(S)$ és equivalent
a la convergència de mesures en la mètrica de Prokhorov.
Per tant, la mètrica de Prokhorov metritza la topologia feble en $\mathcal{P}(S)$.
\end{lema}

\begin{prova}
\textcolor{blue}{\cite{Billingsley1999}, pàg 72}
\end{prova}