Sovint es troba a la literatura una definició alternativa de relativitat compacta
\footnote{Veure \cite[pàg 57]{Billingsley1999} o \cite[def. 4.6]{KaratzasShreve1991}.}.
En aquest treball, hem elavorat una demostració de l'equivalència
entre la definició anterior i la caracterització habitual.
Per a aquest propòsit, necessitem introduir la mètrica de Prokhorov
i establir la seva relació amb la convergència feble de mesures de probabilitat.

\begin{defi}
Diem que un conjunt A és \emph{relativament compacte} si
la seva adherència $\overline{A}$ és compacta.
\end{defi}


% Equivalència definicions
\begin{prova} Primer, suposem que $\Pi$ és relativament compacte
i sigui $\{\P_n\}_n \subset \Pi$ una successió qualsevol.
Volem veure que admet una subsuccessió feblement convergent.

Com que $\Pi$ és relativament compacte, per definició la seva adherència $\overline{\Pi}$ és compacta.
Ja hem vist en el lema \ref{prop-metrica-prokhorov} que $\mathcal{P}(S)$ és un espai mètric
i que la convergència en la mètrica de Prokhorov és equivalent a la convergència feble.
Per tant, $\overline{\Pi}$ és compacte per successions,
i.e. tota successió en $\overline{\Pi}$ admet una subsuccessió feblement convergent.
En particular, la successió $\{\P_n\}_n \subset \Pi \subset \overline{\Pi}$
admet una subsuccessió $\{\P_{n_k}\}_k$ feblement convergent
cap a una mesura de probabilitat $\P \in \overline{\Pi} \subset \mathcal{P}(S)$.

Recíprocament, suposem que tota successió d'elements
de $\Pi$ admet una subsuccessió feblement convergent.
Per demostrar que $\Pi$ és relativament compacte, hem de
veure que la seva adherència $\overline{\Pi}$ és compacta.
Com que per el lema \ref{prop-metrica-prokhorov}, $\mathcal{P}(S)$ és un espai
mètric, hem de veure que $\overline{\Pi}$ és compacte per successions.
És a dir, hem de veure que tota successió en $\overline{\Pi}$
admet una subsuccessió feblement convergent.
Novament per el lema \ref{prop-metrica-prokhorov},
això és equivalent a veure que tota successió en $\overline{\Pi}$
admet una subsuccessió convergent en la mètrica de Prokhorov.

Sigui doncs $\{\P_n\}_n \subset \overline{\Pi}$ una successió qualsevol.
Per la definició d'adherència, per a cada $n \in \mathbb{N}$
existeix una successió $\left\{\P_{n_k}\right\}_k \subset \Pi$
tal que $\P_{n_k} \to \P_n$ quan $m \to \infty$.
Considerem doncs la família de probabilitats
$\left\{ \P_{n_m} : n,m \in \mathbb{N}  \wedge  \P_{n_k} \to \P_n \right\}$
i la seva successió diagonal $\{\P_{n_n}\}_n \subset \Pi$.
Per hipòtesi, aquesta successió admet una subsuccessió
$\{\P_{n_{n_k}}\}_k$ que convergeix cap a una mesura de probabilitat
$\P \in \mathcal{P}(S)$, tal i com voliem demostrar.
\end{prova}