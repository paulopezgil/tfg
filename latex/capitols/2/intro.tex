Per a desenvolupar els resultats que ens permetran aproximar el \emph{moviment Brownià},
treballarem en el marc de l'espai $C[0, \infty)$ de funcions contínues definides a l'interval $[0, \infty)$ i amb valors reals.
En aquest espai, considerarem la mètrica següent:

\begin{equation}
\rho(f,g) \defeq \sum_{n=1}^{\infty} \frac{1}{2^n} \max_{0 \leq t \leq n}\left( \min\left( |f(t) - g(t)|, 1 \right) \right).
\end{equation}

Aquesta aproximació es fonamenta en el \emph{principi d'invariància de Donsker}, també conegut com a \emph{teorema de Donsker} o \emph{teorema central del límit funcional}, que estableix la convergència de certs processos estocàstics discrets cap al moviment Brownià.