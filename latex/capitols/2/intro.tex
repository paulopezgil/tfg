D'ara endavant, treballarem en un espai mètric $(S, \rho)$ separable i complet,
junt amb la seva $\sigma$-àlgebra de Borel $\mathcal{S}$, i denotarem per
$\mathcal{P}(S)$ l'espai de mesures de probabilitat en $(S, \mathcal{S})$.
En el cas particular del moviment brownià, és útil treballar amb l'espai $S=C$
de funcions contínues definides a l'interval $[0, \infty)$ i amb valors reals,
en el qual és possible definir una mètrica
\footnote{Vegeu \cite{KaratzasShreve1991} - pàg 60.},
junt amb la seva $\sigma$-àlgebra de Borel $\mathcal{C}$.


A més, és habitual dotar $\mathcal{P}(S)$ amb l'anomenada \emph{topologia feble}
\footnote{Vegeu \cite{Dudley2002} - pàg 194 o \cite{Bardina2015} - pàg 29.},
i metritzar-la mitjançant la \emph{mètrica de Prokhorov},
per la qual la convergència de mesures de probabilitat
és equivalent a la convergència feble que veurem a continuació
\footnote{Vegeu \cite{Billingsley1999} - pàg 72 o \cite{Dudley2002} - cap. 11.3}.
Si més no, aquests resultats no seran necessaris per al treball que desenvoluparem.