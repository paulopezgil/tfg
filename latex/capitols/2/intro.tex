En aquest capítol estudiarem les condicions sota les quals
certes successions de processos estocàstics convergeixen cap al moviment brownià.
El resultat principal que desenvoluparem és l'anomenat \emph{principi d'invariància de Donsker}, també conegut com a \emph{teorema de Donsker} o \emph{teorema central del límit funcional}.

D'ara endavant, treballarem en un espai mètric $(S, \rho)$ separable i complet,
junt amb la seva $\sigma$-àlgebra de Borel $\mathcal{B}(S)$, i denotarem per
$\mathcal{P}(S)$ l'espai de mesures de probabilitat en $(S, \mathcal{B}(S))$.
En el cas particular del moviment brownià, és útil treballar amb l'espai $S=C[0,\infty]$
de funcions contínues definides a l'interval $[0, \infty)$ i amb valors reals,
en el qual és possible definir una mètrica
\footnote{Vegeu \cite{KaratzasShreve1991} - pàg 60.},
junt amb la seva $\sigma$-àlgebra de Borel $\mathcal{B}(C[0,\infty])$.
\footnote{A més, és habitual dotar $\mathcal{P}(S)$ amb l'anomenada \emph{topologia feble}
i metritzar-la mitjançant la \emph{mètrica de Prokhorov},
per la qual la convergència de mesures de probabilitat
és equivalent a la convergència feble que veurem a continuació.
Si més no, aquests resultats no seran necessaris per al treball que desenvoluparem.}