\section{Convergència feble i convergència en llei}

\begin{defi} % 2.1.1
Sigui $\left\{\P_n\right\}_n \subset \mathcal{P}(S)$ una successió de mesures de probabilitat
i $\P \in \mathcal{P}(S)$ una altra mesura de probabilitat.
Diem que $\left\{\P_n\right\}_n$ \emph{convergeix feblement} cap a $\P$, i escrivim $\P_n \xrightarrow{w} \P$,
si per a tota funció $f: S \to \mathbb{R}$ contínua i acotada es compleix que
\begin{equation} \label{eq-convergencia-feble}
\lim_{n \to \infty} \int_S f(s) d\P_n(s) = \int_S f(s) d\P(s).
\end{equation}
\end{defi}

% \begin{obs} \label{obs-topologia-feble}
% Amb la definició \ref{def-topologia-feble} de topologia feble,
% totes les integrals involucrades en el límit
% (\ref{eq-convergencia-feble}) són funcions contínues.
% \end{obs}

\begin{obs} \label{obs-limit-feble}
En particular, el límit feble $\P$ és una mesura de probabilitat,
ja que si prenem $f \equiv 1$ obtenim que
\begin{equation*}
\lim_{n \to \infty} \int_S 1 \, d\P_n(s) = \int_S 1 \, d\P(s) \implies \lim_{n \to \infty} \P_n(S) = \P(S) \implies \P(S) = 1.
\end{equation*}
\end{obs}



Ara, considerem una variable aleatòria $X$ en un espai de probabilitat $(\Omega, \mathcal{F}, \P)$ amb valors a un espai mesurable $(S, \mathcal{S})$.
Recordem que \emph{la llei de} $X$ es defineix com la mesura de probabilitat $\P X^{-1}$ definida per
\begin{equation*}
\P X^{-1}(B) \defeq \P(X^{-1}(B)) = \P(X \in B) \quad \forall B \in \mathcal{S}.
\end{equation*}





\begin{defi} % 2.1.3
Sigui ${\left\{ (\Omega_n, \mathcal{F}_n, \P_n) \right\}}_n$ una successió d'espais de probabilitat,
on $\P_n \in \mathcal{P}(S)$ $\forall n \in \mathbb{N}$,
i considerem en cada un d'ells una variable aleatòria $X_n$ amb valors a $(S, \mathcal{B}(S))$.
Sigui $\left( \Omega, \mathcal{F}, \\P \right)$ un altre espai de probabilitat, on $\P \in \mathcal{P}(S)$,
i sigui $X$ una variable aleatòria en aquest espai amb valors a $(S, \mathcal{B}(S))$.
Diem que la successió $\{X_n\}_n$ \emph{convergeix en llei} cap a $X$, i escrivim $X_n \xrightarrow{\mathcal{L}} X$,
si les lleis de les variables aleatòries $X_n$ convergeixen feblement cap a la llei de $X$, és a dir, si $\P X_n^{-1} \xrightarrow{w} \P X^{-1}$.
\end{defi}

\begin{obs}
Aquesta definició és equivalent a dir que per a tota funció $f: S \to \mathbb{R}$ contínua i acotada es compleix que
\begin{equation*}
\lim_{n \to \infty} \mathbb{E}[f(X_n)] = \mathbb{E}[f(X)],
\end{equation*}
ja que mitjançant un canvi de variable tenim que
\begin{equation*}
\mathbb{E}[f(X_n)] = \int_{\Omega_n} f(X_n(\omega)) d\P_n(\omega) = \int_S f(s) d\P X_n^{-1}(s),
\end{equation*}
i de manera anàloga,
\begin{equation*}
\mathbb{E}[f(X)] = \int_{\Omega} f(X(\omega)) d\P(\omega) = \int_S f(s) d\P X^{-1}(s).
\end{equation*}
\end{obs}