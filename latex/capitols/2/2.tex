\section{Ajustament i Teorema de Prohorov}

\begin{defi}
Diem que una família $\Pi \subset \mathcal{P}(S)$ de mesures de probabilitat
és relativament compacte si i només si tota successió
d'elements de $\Pi$ admet una subsuccessió feblement convergent.
És a dir, si donada $\{\P_n\}_n \subset \Pi$,
existeixen una subsuccessió $\{\P_{n_k}\}_k \subset \{\P_n\}_n$
i una mesura de probabilitat $\P \in \mathcal{P}(S)$
tals que $\P_{n_k} \xrightarrow{w} \P$.
\end{defi}

\begin{obs} \label{obs-relativament-compacte}
Si una successió $\left\{\P_n\right\}_n \subset \mathcal{P}(S)$ de mesures de probabilitat
convergeix feblement cap a una mesura de probabilitat $\P$,
aleshores és relativament compacte, ja que tota subsuccessió
$\left\{\P_{n_k}\right\}_k$ convergeix feblement o bé cap a $\P$
o bé cap a una altra mesura de probabilitat $\Q \in \left\{\P_n\right\}_n$,
en el que cas que $\P_{n_k} = \Q$ per a tot $k > n_0$, per algun $n_0 \in \mathbb{N}$.
\end{obs}

\begin{defi}
Diem que una família $\Pi \subset \mathcal{P}(S)$ de mesures de probabilitat
és \emph{ajustada} si $\forall \varepsilon > 0$
existeix un conjunt compacte $K_{\varepsilon} \subset S$
tal que $\P(K_{\varepsilon}) \geq 1 - \varepsilon$, $\forall \P \in \Pi$.
\end{defi}

\begin{teo} (Teorema de Prohorov)
Una família $\Pi \subset \mathcal{P}(S)$ de mesures de probabilitat
és relativament compacte si i només si és ajustada.
\end{teo}

\begin{prova}
\textcolor{blue}{M'agradaria comentar si podem evitar aquesta demostració.}
\end{prova}