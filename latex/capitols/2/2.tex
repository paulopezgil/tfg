\section{Ajustament i Teorema de Prohorov}

\begin{defi}
Diem que una familia $\Pi \subset \mathcal{P}(S)$ de mesures de probabilitat
és relativament compacte si i només si tota successió
d'elements de $\Pi$ admet una subsuccessió feblement convergent.
És a dir, si donada $\{\P_n\}_n \subset \Pi$,
existeix una subsuccessió $\{\P_{n_k}\}_k \subset \{\P_n\}_n$
i una mesura de probabilitat $\P \in \mathcal{P}(S)$
tals que $\P_{n_k} \xrightarrow{w} \P$.
\end{defi}

\begin{defi}
Diem que una familia $\Pi \subset \mathcal{P}(S)$ de mesures de probabilitat
és \emph{ajustada} si per a tot $\varepsilon > 0$
existeix un conjunt compacte $K_{\varepsilon} \subset S$
tal que $\P(K_{\varepsilon}) \geq 1 - \varepsilon$ per a tota $\P \in \Pi$.
\end{defi}

\begin{teo} (Teorema de Prohorov)
Una familia $\Pi \subset \mathcal{P}(S)$ de mesures de probabilitat
és relativament compacte si i només si és ajustada.
\end{teo}

\begin{prova}
\textcolor{blue}{M'agradaria comentar si podem evitar aquesta demostració.}
\end{prova}