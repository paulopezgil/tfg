\section{L'espai $C[0, \infty]$}

En el cas particular del moviment brownià, és útil treballar amb l'espai $C[0, \infty)$
de funcions contínues definides a l'interval $[0, \infty)$ i amb valors reals.

\begin{prop} (\cite{KaratzasShreve1991}, exercici 4.1)
La funció $\rho : C[0, \infty) \rightarrow [0, \infty)$ definida per
\begin{equation*}
\rho(f,g) \defeq \sum_{n=1}^{\infty} \frac{1}{2^n} \max_{0 \leq t \leq n}\left( \min\left( |f(t) - g(t)|, 1 \right) \right).
\end{equation*}
és una mètrica a $C[0, \infty)$.
\end{prop}

\textit{Prova:}

A continuació, donarem una caracterització de l'ajustament en l'espai $C[0, \infty)$.
Per a fer-ho, necessitarem la noció de \emph{mòdul de continuïtat} i el \emph{Teorema d'Arzelà-Ascoli}.

\begin{defi}
El \emph{mòdul de continuïtat} és la funció $m: C[0, \infty) \times [0, \infty) \rightarrow [0, \infty)$
definida per
\begin{equation*}
m(f, \delta) \defeq \sup_{|s-t| \leq \delta} |f(s) - f(t)|, \quad \delta \geq 0.
\end{equation*}
\end{defi}

\begin{teo} (Teorema d'Arzelà-Ascoli)
Un conjunt $K \subset C[0, \infty)$ .

\end{teo}