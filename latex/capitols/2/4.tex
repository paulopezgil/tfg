\section[El principi d'invariància]{Convergència en llei cap al moviment Brownià}

Sigui $\{\xi_n\}_n$ una seqüència de variables aleatòries
independents i idènticament distribuïdes amb esperança 0
i variància $\sigma^2$, $0 < \sigma < \infty$. Considerem la successió
de sumes parcials $S_0 = 0$ i $S_n = \sum_{i=1}^n \xi_i$ per a $n \geq 1$.
Per a cada $n \in \mathbb{N}$, considerem el procés de sumes parcials
\begin{equation*}
\frac{1}{\sigma \sqrt{n}} S_{\lfloor nt \rfloor}, \quad t \in [0,\infty],
\end{equation*}
i la corresponent interpolació lineal d'aquest procés,
\begin{equation} \label {interpolacio-lineal}
X_t^{(n)} = \frac{1}{\sigma \sqrt{n}} \left( S_{\lfloor nt \rfloor} + (nt - \lfloor nt \rfloor) \xi_{\lfloor nt \rfloor + 1} \right), \quad t \in [0,\infty].
\end{equation}

En aquest capítol, demostrarem que la successió de processos
$\{X^{(n)}\}_n$ convergeix en llei cap al moviment Brownià estàndard $B$.

\begin{lema}
Siguin $\left\{ X^{n} \right\}_n$, $\left\{ Y^{n} \right\}_n$ i $X$
variables aleatòries definides en un mateix espai de probabilitat
$\left( \Omega, \mathcal{F}, \P \right)$ i amb valors a $\mathcal{S}$.
Suposem que $X^{n} \xrightarrow{\mathcal{L}} X$
i que $\rho(X^{n}, Y^{n}) \xrightarrow{P} 0$.
Aleshores, $Y^{n} \xrightarrow{\mathcal{L}} X$.
\begin{prova}
\textcolor{blue}{\cite{KaratzasShreve1991}, problema 4.16}
\end{prova}
\end{lema}


\begin{lema}
Siguin $\left\{ X^{n} \right\}_n$ i $X$ variables aleatòries
amb valors a un espai mètric $\left(S_1, \rho_1\right)$.
Sigui $\left(S_2, \rho_2\right)$ un altre espai mètric
i sigui $f: S_1 \to S_2$ una funció contínua.
Si $X^{n} \xrightarrow{\mathcal{L}} X$, aleshores
$f(X^{n}) \xrightarrow{\mathcal{L}} f(X)$.
\begin{prova}
\textcolor{blue}{\cite{KaratzasShreve1991}, problema 4.5}
\end{prova}
\end{lema}


\begin{teo} \label{convergencia-interpolacio}
Sigui $\left\{X^{(n)}\right\}_n$ la successió de processos
definits a l'equació \eqref{interpolacio-lineal}.
Siguin $0 \leq t_1 < t_2 < \ldots < t_d < \infty$.
Aleshores,
\begin{equation*}
\left( X_{t_1}^{(n)}, X_{t_2}^{(n)}, \ldots, X_{t_d}^{(n)} \right)
\xrightarrow{\mathcal{L}}
\left( B_{t_1}, B_{t_2}, \ldots, B_{t_d} \right)
\quad \text{quan } n \to \infty,
\end{equation*}
on $\left\{ B_t, \mathcal{F}_t^B; t \geq 0 \right\}$ és un moviment Brownià estàndard.
\begin{prova}
\textcolor{blue}{\cite{KaratzasShreve1991}, teorema 4.15, pàg 67}
\end{prova}
\end{teo}

\begin{teo} \label{convergencia-finita}
Sigui $\left\{ X^{(n)}\right\}_n$ una seqüència ajustada de processos continus
tal que per a qualssevol $0 \leq t_1 \leq ... \leq t_d \leq \infty$,
la seqüència de vectors aleatòris
$\{( X_{t_1}^{(n)},...,X_{t_d}^{(n)})\}_n$
convergeix en llei cap a un vector aleatori
$\left( X_{t_1},...,X_{t_d} \right)$. És a dir,
\begin{equation*}
( X_{t_1}^{(n)},...,X_{t_d}^{(n)}) \xrightarrow{\mathcal{L}} (X_{t_1},...,X_{t_d})
\end{equation*}
Aleshores, $\{X^{(n)}\}_n$
\end{teo}